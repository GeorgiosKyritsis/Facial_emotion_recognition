In this project, we compare different Convolutional Neural Networks on the FER+ dataset. In every architecture we experiment with many different sets of hyper-parameters until we get the best accuracy on the validation set. We show that for the case of a small dataset usually a shallower and wider network perform better than its deeper counterparts. Moreover, the fact that our dataset consists of grayscale images has a negative impact on the performance of transfer learning. Finally we propose a novel algorithm that provides state-of-the-art results on emotion recognition. The key idea is that, two models based on actions units can help the deep neural network to generalise better on unseen data. 

\afterpage{\blankpage}